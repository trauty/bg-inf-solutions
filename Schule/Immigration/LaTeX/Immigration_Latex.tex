\documentclass[12pt]{article}
\usepackage[utf8]{inputenc}
\usepackage[T1]{fontenc}
\usepackage{lmodern}
\usepackage{svg}
\usepackage{hyperref}

\title{Immigration in the United States}
\author{Maximilian Trautwein}

\begin{document}
	
	\maketitle
	
	\begin{figure}[h]
		\centering
		\includegraphics[width=\textwidth]{FBS-Logo_2021-1}
	\end{figure}
	
	\newpage
	\tableofcontents
	
	\newpage
	
	\section{Native Americans}
	The Native Americans were the first settler in America. They came to America over an ice land bridge that connected northern Siberia and todays Alaska. Most of them came from Asia and Siberia and stayed in America after the land bridge melted. They are very diverse, because they are split up into more than 576 individual federally recognized tribes across America.
	
	\section{European Discovery}
	After the Ottoman Turks conquered the Byzantine Empire, they imposed high tax fees on spices from India. Christopher Columbus, an explorer from Italia, visited the Spanish King and Queen to convince them to find another way to India, by sailing around the world. At this point in time many people still believed that the earth was flat, so it was an outrageous idea from Columbus to find a way by travelling around the globe. He convinced the King and Queen and so he sailed with his crew west until they found the first islands in 1492, where they found the first Native Americans. Because the Natives were poorly armed and weren’t as technologically advanced as the Europeans, they succumbed to their weapons and the diseases their bodies weren’t prepared for, especially smallpox. Approximately 19 million Natives died which is 95 per cent of their whole population. As Columbus arrived in Spain with some Natives he brought with him, word began quickly spreading about the new continent. 
	
	\section{First European Settlers}
	The first successful colony was founded by English settlers, and it was Jamestown, Virginia. After 1620, many new immigrants started coming from England, Germany, and Netherlands because of religious prosecution, escaping poverty and wanting a new life or just adventurers, who wanted a fresh breath of air. Many colonies started to grow, especially New Amsterdam, which will be New York later. Most of these new immigrants became farmers because they could only survive on this new continent with food and because there was no infrastructure yet, they needed to do the most necessary and essential things by themselves. Many farmers needed more workers, so the birth rate exploded. Their children essentially helped them on the farm whilst also being the families’ heirs. One great advantage that these widespread and open colonies had, was that diseases could not spread as easily. This resulted in very low death rates, which also contributed to the population increasing. 
	
	\section{Slaves in America}
	The British found tobacco to be a very valuable cash crop, so they started to take or buy people from Africa as slaves as workers. Many families were torn apart, and Africans sold their own people to go and work in America. Many were overworked and died on the farms.
	
	\section{Immigration after founding of the United States}
	There was only little immigration after the US became independent from the British because the US was not stable after this, and many did not want to take the risk. After 1830, a new wave of immigrants came from Europe, mainly from England, Germany, and Ireland because of the cheap farmland that the western expansion had unlocked. Many people were poor in these countries, so the US was a place to restart their lives. The industrial revolution also brought many new job opportunities with it. Most of the immigrants were young men that were between 18 to 25 years old, the perfect workers in their prime phase. The immigrants were split up in their main work fields, Irish worked on infrastructure like train tracks, whilst Asian people were mostly small labor workers, that were picked up at the haven by a contractor after their arrival. Half of the German immigrants became farmers and the other half started smithing and woodworking. Because of these reasons many immigrants arrived at bays in the US, so many in fact, that the population rose over 20 million in 1850. Each decade this number tripled. 
	\newpage
	
	\section{Industrial Revolution intensifies}
	Many new inventions were created, and the industrial revolution demanded more workers than ever before. One of the most influential pieces of creation were the steam-powered ships, that could transport people or goods cheaper, faster, and safer than ever before. Because of this new acquired mobility, it was possible to cross the Atlantic Ocean much easier and come to the US. This new wave of immigrants mainly originated from Europe, especially Italia, Poland, Greece, Sweden, Norway, Hungary, and Libya. The industrial revolution also increased the job opportunities because many new factories were built and these needed workers. These workers helped the industry substituted the workers in the first World War.
	
	\section{First anti-immigrant measures}
	Because of past immigration waves, the US became quite densely populated, especially around and inside of cities. Then the Great Famine hit Ireland in 1845 and most of the crops like potatoes died because of a fungus quickly spreading and killing them. There were over 2 million immigrants from Ireland. These massive waves of immigrants were being looked down upon because the US started to feel like a nation after they declared their independence after the war. Because of this massive “invasion” of their country, the US declared their first anti-immigration law, the Chinese Exclusion Act in 1882. This law banned all Asian contract workers and criminals to ever lay a foot on US soil. Following that, the US also only allowed a certain per cent of people into the US in 1921 with the Emergency Quota Act. This and the stricter quota and the first border patrol in 1924 was followed by an extreme decrease in immigrants. In 1921 immigrants that were granted access to US grounds were more than halved from 800000 to 310000. Illegal immigrants were deported. The only exceptions to this law were people that fled from WW2 and contract workers (except Asian people). 
	\newpage
	
	\section{Current state of immigration in the US}
	Most of the restrictions on new immigrants were lifted piece by piece, starting with the Hart-Celler Act by Emanuel Celler lifted the quotas on accepted immigrants, also removing the Chinese Exclusion Act and shifting of preference to immigrants with highly valuable jobs or jobs that are in high demand. Lifting these restrictions did not fix the issue with illegal immigrants coming from the South. So, they tried again with the Immigration Reform and Control Act in 1986 by making it illegal to hire people that are illegal immigrants or people that have fraudulent documents. The Immigration Act in 1990 that was signed by George W. Bush greatly increased the amount of people allowed into the country (675.000 per year till 1994) and enabled people without proper jobs to be also included. The system was also greatly revised. All of this did not fix the problem of illegal immigrants coming from the South. Trump tried to pull out his trump card but also failed, by building a “wall” (more like a fence) between Mexico and the US. As of 2015, 47 million immigrants lived in the US. Today 20 per cent of all immigrants across the world are living in the US.
	
\end{document}